\documentclass[12pt, a4paper]{article}
\usepackage[nottoc,numbib]{tocbibind}
\usepackage[T1]{fontenc}
\usepackage{tgtermes}
\usepackage{subfiles}
\usepackage{graphicx}
\usepackage[hyphens,spaces]{url}


\linespread{1.25}
\newtheorem{definition}{Definition}


\begin{document}
  \subfile{sections/titlepages}

  \tableofcontents
  \newpage

  \section{Introduction}
  \newpage

  \subfile{sections/ml-and-nlp}

  \subsection{Fake News}
  Fake news is not simply a product of the last decades. It has been part of humanity for at least centuries. Its first significant outburst dates back to 1400s, when the printing press was invented. Ever since then, fabricated news have been more and more weaponized to manipulate people at increasingly larger scales. In 1835, The New York Sun published 6 articles presenting the discovery of life on the moon. Both World Wars were heavily marked by propaganda, and false and misleading news \cite{a4}. And more recently, the US 2016 presidential elections and the COVID-19 pandemic were subjects of large amounts of fake news.

  Despite being so widespread, there is no globally agreed definition of the "fake news" term. Nonetheless, many sources provide definitions which have two primary elements in common: authenticity and intent. As for authenticity, fake news contain claims that are factually and verifiably false. As for intent, the major purpose of fake news is to deceive the consumers \cite{a2}. That being said, a concise definition of fake news could be formulated in the following manner:

  \begin{definition}
    By fake news we mean any news item that is deliberately and factually untrue and misleading.
  \end{definition}

  \subsection{Fake News Detection}
  Fake News Detection represents a new area of study. On the one hand, it emerged thanks to the continuous advancements in Artificial Intelligence and, specifically, Deep Learning. On the other hand, it appeared as virtually a necessity, given the recent escalation of the fake news phenomenon, on account of the simultaneous raise of digital media.

  Digital and social media are environments in which the dissemination of information is more facile and rapid than at any point in history. Fabricated news have been heavily published and shared as a means of deception, on subjects of high impact, such as the 2016 US presidential elections and the COVID-19 pandemic.

  At present, there are multiple fake news identification resources. They each contribute in various manners to validating the veracity of news and debunking rumours, hoaxes, conspiracies.


  \subsection{Related Work}
  \newpage

  \subfile{sections/application-development}


  \section{Conclusion and Future Work}
  \newpage

  \bibliographystyle{unsrt}
  \bibliography{sources}
  \nocite{*}
\end{document}